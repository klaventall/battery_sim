\newcommand{\ones}{\mathbf 1}
\newcommand{\reals}{{\mbox{\bf R}}}
\newcommand{\integers}{{\mbox{\bf Z}}}
\newcommand{\symm}{{\mbox{\bf S}}} % symmetric matrices

\newcommand{\nullspace}{{\mathcal N}}
\newcommand{\range}{{\mathcal R}}
\newcommand{\Rank}{\mathop{\bf Rank}}
\newcommand{\Tr}{\mathop{\bf Tr}}
\newcommand{\diag}{\mathop{\bf diag}}
\newcommand{\card}{\mathop{\bf card}}
\newcommand{\rank}{\mathop{\bf rank}}
\newcommand{\conv}{\mathop{\bf conv}}
\newcommand{\prox}{\mathbf{prox}}

\newcommand{\Expect}{\mathop{\bf E{}}}
\newcommand{\Prob}{\mathop{\bf Prob}}
\newcommand{\Co}{{\mathop {\bf Co}}} % convex hull
\newcommand{\dist}{\mathop{\bf dist{}}}
\newcommand{\argmin}{\mathop{\rm argmin}}
\newcommand{\argmax}{\mathop{\rm argmax}}
\newcommand{\epi}{\mathop{\bf epi}} % epigraph
\newcommand{\Vol}{\mathop{\bf vol}}
\newcommand{\dom}{\mathop{\bf dom}} % domain
\newcommand{\intr}{\mathop{\bf int}}
\newcommand{\sign}{\mathop{\bf sign}}

\newcommand{\cf}{{\it cf.}}
\newcommand{\eg}{{\it e.g.}}
\newcommand{\ie}{{\it i.e.}}
\newcommand{\etc}{{\it etc.}}

\title{Optimal Battery Load Scheduling for Minimizing Energy Costs}
\author{
Katie Wurman
}
\date{\today}

\documentclass[12pt]{article}
\usepackage{float}
\usepackage{amssymb,amsmath}
\usepackage{graphicx}
\graphicspath{ {figures/} }


\begin{document}
\maketitle

\section{Introduction}
In this project we will simulate a dynamic battery system using real load and solar generation data. The objective is to design a battery load schedule which minimizes total energy costs. Energy costs vary over time and are proportional to the load. The approach in this paper is to simulate the battery as a linear dynamical system with a control input which dictates how much load the battery should absorb or release. The control signal will be computed using an optimal control law (a.k.a model predictive control), where our objective is cost minimization and our constraints are the battery dynamics. Results of this approach provide a battery load schedule which reduce energy costs by 7.3\%.

\section{Problem Statement}\label{problem statement}
\paragraph{Billing Schedule}
Energy costs are subject to a time-of-use billing periods and account not just for the quantity of energy consumed but also for the cost to deliver that energy (know as a demand charge). There are three time-of-use periods which affect pricing: peak, part-peak, and off peak. Each period has an energy charge and a demand charge. The energy charge is applied to the total load within the time-of use period. The demand charge is applied to the highest load value within the time-of-use period. A mathematical definition of the energy costs is as follows:

Define the load data as a vector $v \in \reals^N$ where $N$ is the number of load measurements over the billing period. In this model, measurements are taken every fifteen minutes and the billing period is one week. Define $\alpha_{pk}, \alpha_{pp}, \alpha_{off}$ as the scale factors for the energy cost. Similarly, define $\beta_{pk}, \beta_{pp}, \beta_{all}$ as the scale factors for the demand cost. Then the total energy costs is defined as
\begin{equation}
\begin{split}
J(v) &= \sum_{k=1}^N \big( \alpha_{pk} I_{pk}(k)v_k + \alpha_{pp} I_{pp}(k)v_k+ \alpha_{off} I_{off}(k)v_k \big) \\
&+ \beta_{pk} \max_{k=1,2..,N} I_{pk}(k)v_k +\beta_{pk} \max_{k=1,2..,N} I_{pp}(k)v_k +\beta_{all} \max_{k=1,2..,N} v_k
\end{split}
\end{equation}
where
\begin{equation}\label{costfun}
I_{pk}(k)=
\begin{cases}
1,& \text{if } k \text{is in the peak period}\\
0, & \text{otherwise}
\end{cases}
\end{equation}
The indicator functions$ I_{pp}{k}$ and $I_{off}(k)$ are defined similarly.



\paragraph{Battery Dynamics}
We assume the batter has simple linear dynamics without energy loss. The parameters $\epsilon_{acdc}$ and $\epsilon_{dcac}$ denote the input and output inefficiencies of the battery and we set a maximum output of $P_{max}$ and a maximum capacity of $C_{max}$ on the battery. Denote $s(t)$ as the charge of the battery at time $t$ and denote $u(t)$ as our control input. When $u(t)$ is positive, the battery charges and when $u(t)$ is negative the battery discharges. Then for each time $t \in \{1,...,N\}$ we evolve the system as
\begin{equation}
\begin{split}
s(t+1) &= s(t) + \epsilon_{acdc} u(t) \\
y(t) &= \epsilon_{dcac} u(t) \\
s(t) & \leq C_{max} \quad t \in \{1,...,N\} \\
u(t) & \leq P_{max} \quad t \in \{1,...,N\}
\end{split}
\end{equation}

\paragraph{Optimal Scheduling}
Next, we will define the control law which will compute the optimal load schedule for the battery in order to minimize costs. We will define this control law as an optimal control problem where we select a set of inputs $u(0),...,u(N-1)$ which will minimize our objective function $J(\cdot)$. Note that without the battery, our cost is $J(v)$. If we include the battery our cost is $J(v+y)$. Using $u \in \reals^{N-1} $ and $s \in \reals^{N}$ as our variables and $v$ as our load data, we solve the optimization problem
\begin{quote}
\begin{equation}\label{opteq}
\begin{array}{ll}
\mbox{minimize} & J(v + \epsilon_{dcac} u) \\
\mbox{subject to} & s(t+1) = s(t) + \epsilon_{acdc} u(t) \quad t \in \{1,...,N\} \\
	& s(N) = 0 \\
	& s(t) \geq 0 \\
	& s(t) \leq C_{max} \\
	& u(t) \geq -P_{max} \\\
	& u(t) \leq P_{max} \\
	& v(t) + \epsilon_{dcac} u(t) \geq 0
\end{array}
\end{equation}
The last constraint in this problem $v(t) + \epsilon_{dcac} u(t) \geq 0$ ensures that we are only purchasing from the grid and not selling.
\end{quote}


\section{Implementation}
To simulate this system and compute the optimal schedule, a Python application called \texttt{battery\_sim} was built utilizing the \texttt{cvxpy} package, which solves convex optimization problems using disciplined convex programming.

\paragraph{The Data}
The simulation was generated using the provided load and solar data. To compute the net load post solar data, the hourly solar data was evenly distributed across 15-min intervals and subtracted from the raw load data. The routine used to generate the load data can be found in \texttt{load\_generator.py}.

\paragraph{Utility Rate Generator}
To characterize the time-of-use charges, an indicator matrix $A \in \reals^{NxN}$ was generated such that $[Ax]_k = 0$ if $k$ is not in the rate period and $[Ax]_k = 1 $ if it is. Four indicator matrices $A_{peak}, A_{part\_peak}, A_{off\_peak}$ and $A_{all\_peak}$ are generated as elements of the \texttt{UtilityRateGenerator} class. The source for this class is located in \texttt{util\_rate\_generator.py}.


\paragraph{Battery Simulator}
Taking as inputs the load \texttt{v} and a utility rate generator \texttt{urg} the battery simulator formulates and solves the optimization problem described in (\ref{opteq}). This simulation and support methods can be found in \texttt{battery\_simulator.py}.

\section{Results}
To begin, let's run the simulation on some artificial load data, so we can get an intuitive understanding of the model. Assuming a constant load of 150kW we see the following results:

\begin{figure}[h]
\includegraphics[width=\textwidth]{basicsummary}
\end{figure}
To see the behaviour more explicitly, let's look at the battery load superimposed on the load scaled by the time-of-use cost factors:
\begin{figure}[h]
\includegraphics[width=\textwidth]{loadVcost}
\end{figure}
The behavior is aligned in a perfect anti-pattern: When the cost is lowest, the battery load is highest, which means it's charging. Then when the cost is highest, the battery load is lowest, which means it's discharging. It works!

Next, let's look at the results of the simulation using the data provided in \texttt{load\_data.csv} and \texttt{generation\_data.csv}:
\begin{figure}[H]
\includegraphics[width=\textwidth]{realdatasummary}
\end{figure}

\begin{figure}[H]
\includegraphics[width=\textwidth]{batVsys}
\end{figure}
Again, we see negative battery loads when the load is highest and positive battery loads when the load is lowest. This is especially apparent on the weekend, when the complicating demand charges do not apply.

\paragraph{Cost Analysis}
Running this system without a battery results in a total energy cost of \$10,805.46. When the battery is included with optimal scheduling the total energy cost drops to \$10,067.35, resulting in 7.3\% savings over the billing period.

\section{Conclusions}\label{conclusions}
For this project we have come up with a model of a dynamic battery system which determines an optimal battery load schedule to minimize total energy costs. The results show a battery schedule which is capable of buying supply when costs are lowest and storing and releasing it when costs are highest. The overall cost savings over a week long billing period were 7.3 \%. Further work on this project will be to build an estimator to predict load demand in the presence of missing data using a Kalman filter and include a more detailed model of battery dynamics which includes battery degradation. 

\bibliographystyle{abbrv}
\bibliography{main}

\end{document}
This is never printed

